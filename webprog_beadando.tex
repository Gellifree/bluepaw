\documentclass[a4paper,12pt]{article}
\usepackage[T1]{fontenc}
\PassOptionsToPackage{defaults=hu-min}{magyar.ldf}
\usepackage[magyar]{babel}
\usepackage[a4paper]{geometry}
\geometry{margin=2cm}


\begin{document}
\title{\textsc{Webprogramozás III. Gy. \\ {\normalsize (LBT\_IM744G2)}} \\ {\normalsize Beadandó feladat dokumentáció}}
\author{Kovács Norbert \\ ANOXWJ}
\maketitle
\newpage
\tableofcontents
\newpage


\section{Weboldal koncepciója}
\subsection{Technikai részletek}
A beadandó feladatot, az órán is alkalmazott \textit{xamp} szerverrel kívánom megvalósítani, \textit{apache netbeans} fejlesztőkörnyezettel. A mysql kezelését a videóban is használt, \textit{phpmyadmin} oldalon intézem. Nem fontos részlet, a megvalósítás szempontjából, de a windows operációs rendszer, virtuális gépként fut.

\begin{itemize}
	\item OS: Windows 10 Pro (21H1)
	\item Kernel: 10.0.19041.1023
	\item Mysql: 10.4.19-MariaDB
	\item PHP: 8.0.6
\end{itemize}

\subsection{Elképzelés}
A weboldal egy \textit{fiktív} szervezet köré épül, aminek a feladata  az \textit{állategészségügy}, és \textit{állatvédelem}. A fiktív szervezetnek több városban lesznek épületei országszerte. Egy városban több különféle épület is tartozhat az adott telephez. Például tartozhat állatmenhely, állatorvosi rendelő, raktárépület stb. Ezeknek az épületeknek vannak hozzájuk tartozó egységek. Egy menhely állatok tárolására alkalmas kennelekkel, egy rendelő/raktár szobákkal rendelkezik. 

\section{Adatbázis tervezése}
\subsection{ER modell}
\subsection{Relációs adatmodell}
\subsection{SQL parancsok}
\subsubsection{Táblák létrehozása}



\end{document}
